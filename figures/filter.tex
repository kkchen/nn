\begin{tikzpicture}[x=1.5mm, y=1.5mm, auto]
    % The image.
    \foreach \depth in {2, ..., 0} {
        \image{-\depth}{\depth}
    }

    % Tires.
    \foreach \i in {5.5, 15.5} {
        \visible<2-3>{\draw [very thick] (\i, -15.5) circle (1.5);}
    }

    % A filter with tires.
    \visible<3>{
        \foreach \depth in {2, ..., 0} {
            \filter{-\depth}{-28 + \depth}
        }

        \draw [very thick] (2.5, -30.5) circle (1.5);
    }

    % Start the convolution.
    \foreach \i in {0, ..., 5} {
        \pgfmathtruncatemacro{\overlay}{\i + 4}

        % Overlay the filter on the image.
        \visible<\overlay>{
            \foreach \depth in {2, ..., 0} {
                \image{-\depth}{\depth}
                \filter{\i - \depth}{\depth}
                \draw [very thick, gray] (\i + 2.5, -2.5) circle (1.5);
            }

            % Tires.
            \foreach \i in {5.5, 15.5} {
                \visible<\overlay>{\draw [very thick] (\i, -15.5) circle (1.5);}
            }
        }

        \visible<\overlay-9>{\featuremap{\i + 4}{-28}{1}}
        \visible<\overlay>{\draw [path] (\i + 2.5, -2.5) -- (\i + 4.5, -28.5);}
    }

    % Complete the feature map.
    \visible<10->{
        \foreach \depth in {2, ..., 0} {
            \image{-\depth}{\depth}
        }

        \featuremap{4}{-28}{16}

        \foreach \i in {5.5, 15.5} {
            \filter{\i - 2.5}{-13} % A filter matching the tire.
            \draw [very thick] (\i, -15.5) circle (1.5); % Tires.

            % Lit-up feature map elements.
            \draw [very thick, fill=green] (\i + 1.5, -41) rectangle (\i + 2.5, -42);
            \draw [path] (\i, -15.5) -- (\i + 2, -41.5);
        }
    }

\end{tikzpicture}
%%% Local Variables:
%%% mode: latex
%%% TeX-master: "../nn"
%%% End:
