\begin{tikzpicture}[node distance=3mm]
    % Box.
    \draw [fill=green!10, rounded corners] (-0.75, -0.25) rectangle (1.75, 1.75);

    % Nodes.
    \node (input 0) [io mini] {};
    \node (input 1) [io mini off, right=of input 0] {};
    \node (input 2) [io mini, right=of input 1] {};

    \node (dense 01) [neuron mini, above=of input 0] {};
    \node (dense 00) [neuron mini off, left=of dense 01] {};
    \node (dense 02) [neuron mini, right=of dense 01] {};
    \node (dense 03) [neuron mini off, right=of dense 02] {};
    \node (dense 04) [neuron mini, right=of dense 03] {};

    \node (dense 10) [neuron mini, above=of dense 00, xshift=2.5mm] {};
    \node (dense 11) [neuron mini, right=of dense 10] {};
    \node (dense 12) [neuron mini, right=of dense 11] {};
    \node (dense 13) [neuron mini off, right=of dense 12] {};

    \foreach \i in {0, 1} {
        \pgfmathtruncatemacro{\j}{\i + 1}
        \node (output \i) [io mini, above=of dense 1\j] {};
    }

    % Connections.
    \foreach \i in {1, 2, 4} {
        \foreach \j in {0, 2} {
            \draw (input \j) -- (dense 0\i);
        }

        \foreach \j in {0, ..., 2} {
            \draw (dense 0\i) -- (dense 1\j);
        }
    }

    \foreach \i in {0, ..., 2} {
        \foreach \j in {0, 1} {
            \draw (dense 1\i) -- (output \j);
        }
    }
\end{tikzpicture}
%%% Local Variables:
%%% mode: latex
%%% TeX-master: "../nn"
%%% End:
