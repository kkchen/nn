\section{Dense neural networks}

\subsection{}

\begin{frame}
    \frametitle{The neuron}

    Consider an input $\x \in \Reals^q$.
    \begin{block}{}
        Consider some \emph{weight vector} $\w \in \Reals^q$ and \emph{bias} $b \in \Reals$.
        The mapping
        \begin{align*}
            g &: \Reals^q \to \Reals \\
            &\hspace{1.25ex} \x \mapsto \w \cdot \x + b
        \end{align*}
        is an \alert{affine transformation}.
    \end{block}
    \pause

    Consider a generally nonlinear \alert{activation function} $\sigma: \Reals \to \Reals$.

    \begin{block}{}
        An (artifical) \alert{neuron} is the function $\phi = \sigma \circ g$, i.e.,
        \begin{align*}
            \phi &: \Reals^q \to \Reals \\
            &\hspace{1.25ex} \x \mapsto \sigma(\w \cdot \x + b)
        \end{align*}
    \end{block}
    \pause

    \begin{itemize}
        \item Historically, artificial neuron modeled on biological neurons
        \begin{itemize}
            \item Bio neuron triggers impulse by nonlinear function of inputs
        \end{itemize}
        \item Today, relation is largely only conceptual/philosophical
    \end{itemize}
\end{frame}

\begin{frame}
    \frametitle{Single hidden-layer neural network: vector equations}

    Consider $n$ neurons each taking in $\x \in \Reals^q$: given
    \begin{itemize}
        \item weight vectors $\w_k \in \Reals^q$, $k = 1, \ldots, n$,
        \item biases $b_k \in \Reals$, $k = 1, \ldots, n$,
    \end{itemize}
    \begin{block}{\alert{Hidden layer}}
        \vspace{-1em}
        \begin{align*}
            z_k &= \sigma(\w_k \cdot \x + b_k), \quad k = 1, \ldots, n \\
            \z &= \begin{bmatrix} z_1 & \cdots & z_n \end{bmatrix} \in \Reals^n
        \end{align*}
    \end{block}
    \pause

    Let the model output $\y \in \Reals^p$ be affine transformations of $\z$: given
    \begin{itemize}
        \item weight vectors $\v_k \in \Reals^n$, $k = 1, \ldots, p$
        \item biases $c_k \in \Reals$, $k = 1, \ldots, p$
    \end{itemize}
    \begin{block}{Single hidden-layer neural network}
        \vspace{-1em}
        \begin{align*}
            y_k &= \v_k \cdot \z + c_k, \quad k = 1, \ldots, p \\
            \y &= \begin{bmatrix} y_1 & \cdots & y_p \end{bmatrix}
        \end{align*}
    \end{block}
\end{frame}

\begin{frame}
    \frametitle{Single hidden-layer neural network: figure}

    \centering
    \begin{tikzpicture}[>=latex, node distance=9mm]
    % Inputs.
    \uncover<+->{
        \node (x1) [scalar] {$x_1$};
        \node (x2) [scalar, right=of x1] {$x_2$};
        \node (x3) [scalar, right=of x2] {$\cdots$};
        \node (x4) [scalar, right=of x3] {$x_q$};

        \node [right=1.25cm of x4] {$\x$};

    }

    \uncover<+->{
        % Hidden layer.
        \draw [very thick, rounded corners, fill=red!10]
        (-1.2, 0.9) rectangle (6.27, 3.5);

        % Hidden layer combinations.
        \node (affine x1) [affine, above=of x1, xshift=-5mm] {affine};
        \node (affine x2) [affine, right=6mm of affine x1] {affine};
        \node (affine x3) [affine, right=6mm of affine x2] {affine};
        \node (affine x4) [affine, right=6mm of affine x3] {$\cdots$};
        \node (affine x5) [affine, right=6mm of affine x4] {affine};

        \node [right=4.3mm of affine x5] {$\w_k \cdot \x + b_k$};
    }

    \uncover<+->{
        % Hidden layer activations.
        \foreach \i in {1, 2, 3, 5} {
            \node (sigma\i) [activation, above=4mm of affine x\i] {$\sigma$};
        }

        \node (sigma4) [activation, above=4mm of affine x4] {$\cdots$};

        \node [right=5.5mm of sigma5] {$z_k = \sigma(\w_k \cdot \x + b_k)$};
    }

    % Output combinations.
    \uncover<+->{
        \node (affine y1) [affine, above=of sigma2] {affine};
        \node (affine y2) [affine, above=of sigma3] {$\cdots$};
        \node (affine y3) [affine, above=of sigma4] {affine};

        \node [right=1.88cm of affine y3] {$y_k = \v_k \cdot \z + c_k$};
    }

    % Outputs.
    \uncover<+->{
        \node (y1) [scalar, above=of affine y1] {$y_1$};
        \node (y2) [scalar, above=of affine y2] {$\cdots$};
        \node (y3) [scalar, above=of affine y3] {$y_p$};

        \node [right=2cm of y3] {$\y$};
    }

    % Connections.

    \foreach \i/\a in {1/130, 2/110, 3/90, 4/70, 5/50} {
        % Inputs to affines.
        \foreach \j/\b in {1/230, 2/257, 3/283, 4/310} {
            \uncover<2->{\draw [path] (x\j.\a) -- (affine x\i.\b);}
        }

        % Hidden layer affines to activations.
        \uncover<3->{\draw [path] (affine x\i) -- (sigma\i);}
    }

    \foreach \j/\b in {1/120, 2/90, 3/60} {
        % Hidden layer activations to output affines.
        \foreach \i/\a in {1/230, 2/250, 3/270, 4/290, 5/310} {
            \uncover<4->{\draw [path] (sigma\i.\b) -- (affine y\j.\a);}
        }

        % Output affines to outputs.
        \uncover<5->{\draw [path] (affine y\j) -- (y\j);}
    }
\end{tikzpicture}

%%% Local Variables:
%%% mode: latex
%%% TeX-master: "../nn"
%%% End:

\end{frame}

\begin{frame}
    \frametitle{Single hidden-layer neural network: matrix equations}

    The equations are more compact in matrix form:
    \begin{itemize}
        \item Weight matrix $\W \in \Reals^{n \times q}$
        \item Bias vector $\b \in \Reals^n$
        \item Element-wise activation function
        \begin{align*}
            \SIGMA &: \Reals^n \to \Reals^n \\
            &\hspace{1.25ex} \begin{bmatrix} \xi_1 & \cdots & \xi_n \end{bmatrix} \mapsto
            \begin{bmatrix} \sigma(\xi_1) & \cdots & \sigma(\xi_n) \end{bmatrix}
        \end{align*}
    \end{itemize}

    \begin{block}{Hidden layer}
        \begin{equation*}
            \z = \SIGMA(\W \x + \b)
        \end{equation*}
    \end{block}
    \pause

    \begin{itemize}
        \item Weight matrix $\V \in \Reals^{q \times n}$
        \item Bias vector $\c \in \Reals^q$
    \end{itemize}

    \begin{block}{Single hidden-layer neural network}
        \begin{equation*}
            \y = \V \z + \c
        \end{equation*}
    \end{block}
\end{frame}
% Nonlinearities: why necessary, and common choices
% Universal approximation

%%% Local Variables:
%%% mode: latex
%%% TeX-master: "../nn"
%%% End:

