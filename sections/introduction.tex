\section{Introduction}

\subsection{}

\begin{frame}
    \frametitle{Resources}

    \begin{itemize}
        \item Ian Goodfellow, Yoshua Bengio, and Aaron Courville.
        \emph{Deep Learning}.
        MIT Press, 2016
        \nocite{GoodfellowDL}
        \begin{itemize}
            \item Excellent primer
            \item \blueurl{http://www.deeplearningbook.org}
        \end{itemize}
        \item These slides: \blueurl{https://github.com/kkchen/nn}
        \item References at the end of these slides
        \item Physics-specific resources
        \begin{itemize}
            \item Search \texttt{https://meetings.aps.org/Meeting/DFD1x/} \texttt{SearchAbstract} (with \texttt{x} = 7, 6, etc.) for ML abstracts
            \item NIPS 2017 Workshop on \bluelink{https://dl4physicalsciences.github.io}{Deep Learning for Physical Sciences}
        \end{itemize}
        \item Talk to me! \smiley
    \end{itemize}
\end{frame}

\begin{frame}
    \frametitle{A super brief history of machine learning}
    \begin{itemize}
        \item Mid-20th century: birth of essential machine learning ideas
        \item 1957: Birth of the \alert{perceptron}, basis of modern neural networks \citep{RosenblattCornell57}
        \item 1970s--1980s: discovery of \alert{backpropagation} to train ML models efficiently \citep{RumelhartNature86}
        \item 1980s: \alert{convolutional} neural networks, \alert{recurrent} neural networks, \alert{reinforcement} learning
        \item 1989: \alert{Universal approximation theorem}: neural networks can model anything nice \citep{CybenkoMCSS89,HornikNN89}
    \end{itemize}
    \pause

    So why haven't we heard much about ML until recently?
    \begin{itemize}
        \item 1970s--1990s: \alert{AI Winter}: ML thought to be intractable and unsexy.
        Rejected by much of the scientific community as useless!
    \end{itemize}
    \pause

    What changed?
    \begin{itemize}
        \item 2010s: emergence of \alert{deep learning} coupled with \alert{big data} $\implies$ ML starts to achieve the impossible
    \end{itemize}
\end{frame}

\begin{frame}
    \frametitle{Machine learning in fluid dynamics?}
    \# APS DFD abstracts with ``machine learning,'' ``neural network,'' or ``deep learning''

    \begin{center}
        \begin{tabular}{r|cccc}
            year & 2014 & 2015 & 2016 & 2018 \\
            \hline
            \# & 8 & 11 & 11 & \alert{28}
        \end{tabular}
    \end{center}
    \pause

    My perspective:

    \begin{itemize}
        \item Fluid dynamics is light years behind stats/CS in ML
        \item ML may or may not lead to a breakthrough in fluids modeling
        \begin{itemize}
            \item ML knowledge is developing at an astronomical rate
            \item The new norm: things thought impossible 10 years ago are now mundane
            \item But: ML doesn't inherently know physics.
            Should we expect ML to beat physics-based models?
            I don't know!
        \end{itemize}
    \end{itemize}
\end{frame}

\begin{frame}
    \frametitle{My advice}

    \begin{itemize}
        \item Pursue ML research in fluids
        \item Or don't, but pay close attention to what others are doing.
        An ML explosion in fluids might or might not happen; if it does, don't miss it.
    \end{itemize}
    \pause

    \begin{block}{What do fluid dynamicists need to do right now?}
        \begin{itemize}
            \item 40\% of the work: catch up to the rest of world on ML
            \item 50\%: figure out \alert{what the right problems are} in the first place
            \item Remaining 10\%: solve them
        \end{itemize}
    \end{block}
    \pause

    Be warned: the ML research community is \emph{extremely} fast-paced
    \begin{itemize}
        \item Most cited papers seem to be $< 2$ years old
        \item Anything older is likely obsolete!
        \item Common for previously unsolved problems to be solved within months; must keep up with conferences \& arXiv
    \end{itemize}
\end{frame}

\begin{frame}
    \frametitle{My objective}

    \begin{itemize}
        \item<+-> Teach a semester's worth of machine learning material in two hours
        \begin{itemize}
            \item Except for \emph{recurrent neural networks}: they're particularly relevant for dynamical systems, so we'll spend two hours on that Thursday
        \end{itemize}
        \item<.-> Obviously I can't go into much detail about anything
        \item<+-> You guide the next two hours:
        \begin{itemize}
            \item Stop me and ask lots of questions if you want \smiley
            \item Or just let me talk, either way
        \end{itemize}
        \item<+-> I will use standard ML terminology/notation, instead of tailoring to an engineering audience
        \begin{itemize}
            \item Otherwise you won't be able to understand the ML literature
        \end{itemize}
    \end{itemize}
\end{frame}

\begin{frame}
    \frametitle{What I will not do}
    \begin{itemize}
        \item<+-> Offer expert knowledge on machine learning in fluids/engineering/physics
        \begin{itemize}
            \item That's because I'm not an expert on that
        \end{itemize}
        \item<+-> Teach how to write ML code
        \begin{itemize}
            \item That would require its own lecture
            \item Today and Thursday will be mostly theory, with some brief examples
            \item $\exists$ some example code for recurrent neural networks on Thursday
        \end{itemize}
    \end{itemize}
\end{frame}

%%% Local Variables:
%%% mode: latex
%%% TeX-master: "../nn"
%%% End:
