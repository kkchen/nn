\section{Introduction}

\subsection{}

\begin{frame}
    \frametitle{Resources}

    \begin{itemize}
        \item Ian Goodfellow, Yoshua Bengio, and Aaron Courville.
        \emph{Deep Learning}.
        MIT Press, 2016
        \nocite{GoodfellowDL}
        \begin{itemize}
            \item Excellent primer
            \item \blueurl{http://www.deeplearningbook.org}
        \end{itemize}
        \item References at the end of these slides
        \item Search \texttt{https://meetings.aps.org/Meeting/DFD1x/} \texttt{SearchAbstract} (with \texttt{x} = 7, 6, etc.) for ML abstracts
        \item NIPS 2017 Workshop on \bluelink{https://dl4physicalsciences.github.io}{Deep Learning for Physical Sciences}
        \item Talk to me! \smiley
    \end{itemize}
\end{frame}

\begin{frame}
    \frametitle{Machine learning in fluid dynamics?}
    \# APS DFD abstracts with ``machine learning,'' ``neural network,'' or ``deep learning''

    \begin{center}
        \begin{tabular}{r|cccc}
            year & 2014 & 2015 & 2016 & 2018 \\
            \hline
            \# & 8 & 11 & 11 & \alert{28}
        \end{tabular}
    \end{center}

    My perspective:

    \begin{itemize}
        \item Fluid dynamics is light years behind stats/CS in ML
        \item ML may or may not lead to a breakthrough in fluids modeling
        \begin{itemize}
            \item ML knowledge is developing at an astronomical rate
            \item The new norm: things thought impossible 10 years ago are now mundane
            \item But: ML doesn't inherently know physics.
            Should we expect ML to beat physics-based models?
            I don't know!
        \end{itemize}
    \end{itemize}
\end{frame}

\begin{frame}
    \frametitle{My advice}

    \begin{itemize}
        \item Pursue ML research in fluids
        \item Or don't, but pay close attention to what others are doing.
        An ML explosion in fluids might or might not happen; if it does, don't miss it.
    \end{itemize}

    \begin{block}{What do fluid dynamicists need to do right now?}
        \begin{itemize}
            \item 40\% of the work: catch up to the rest of world on ML
            \item 50\%: figure out \alert{what the right problems are} in the first place
            \item Remaining 10\%: solve them
        \end{itemize}
    \end{block}
\end{frame}

\begin{frame}
    \frametitle{My objective}

    \begin{itemize}
        \item Teach a semester's worth of machine learning material in two hours
        \begin{itemize}
            \item Except for \emph{recurrent neural networks}: they're particularly relevant for dynamical systems, so we'll spend two hours on that Thursday
            \item Obviously I can't go into much detail about anything
        \end{itemize}
        \item You guide the next two hours:
        \begin{itemize}
            \item Stop me and ask lots of questions if you want \smiley
            \item Or just let me talk, either way
        \end{itemize}
        \item I will use standard ML terminology/notation, instead of tailoring to an engineering audience
        \begin{itemize}
            \item Otherwise you won't be able to understand the ML literature
        \end{itemize}
    \end{itemize}
\end{frame}

\begin{frame}
    \frametitle{What I will not do}
    \begin{itemize}
        \item Offer expert knowledge on machine learning in fluids/engineering/physics
        \begin{itemize}
            \item That's because I'm not an expert on that
        \end{itemize}
        \item Teach how to write ML code
        \begin{itemize}
            \item That would require its own lecture
            \item Today and Thursday will be mostly theory, with some brief examples
            \item I'll have some example code on Thursday for recurrent neural networks
        \end{itemize}
    \end{itemize}
\end{frame}

%%% Local Variables:
%%% mode: latex
%%% TeX-master: "../nn"
%%% End:
