\section[Advice]{General advice}

\subsection{}

\begin{frame}
    \frametitle{Hardware}

    \begin{itemize}
        \item \alert{Use GPUs.}
        The bulk of forward and backprop is matrix--vector multiplies; common for GPUs to be $\O(10 \text{--} 100)\times$ faster
        \item \alert{Use GPUs built for NNs.}
        The NVIDIA Volta and Google tensor processing unit (TPU) are built from ground-up to be lightning fast for NNs (e.g., 16-bit multiply-add for 32-bit result)
        \item \alert{Use mini-batch sizes and widths that are integer powers of 2.}
        Not a hard rule, but GPUs tend to more efficient this way
        \item \alert{Use 32-bit floats.}
        Neural networks are not turbulence simulations; no tangible benefit to 64-bit doubles
    \end{itemize}
\end{frame}

\begin{frame}
    \frametitle{Software}

    \begin{itemize}
        \item $\exists$ many NN platforms; unclear which will win out in the end.
        \item Most popular now: Keras, TensorFlow, PyTorch
        \begin{itemize}
            \item TensorFlow seems to be winning, but list changes rapidly
        \end{itemize}
        \item How do they compare in my opinion?
    \end{itemize}

    \begin{center}
        \begin{tikzpicture}
    \def\pos{0}
    \scale{difficult}{easy}
    \entry{TF}{-3}
    \entry{PT}{-1}
    \entry{K}{3}

    \def\pos{-1.5}
    \scale{limited \\ capability}{flexible}
    \entry{K}{-3}
    \entry{TF}{1.7}
    \entry{PT}{3}

    \def\pos{-3}
    \scale{limited \\ docs}{excellent \\ docs}
    \entry{K}{-1.2}
    \entry{TF}{1.5}
    \entry{PT}{3}

    \def\pos{-4.5}
    \scale{small \\ user base}{large \\ user base}
    \entry{PT}{-1.5}
    \entry{K}{1}
    \entry{TF}{3}
\end{tikzpicture}
%%% Local Variables:
%%% mode: latex
%%% TeX-master: "../nn"
%%% End:

    \end{center}
\end{frame}

\begin{frame}
    \frametitle{Neural network architecture}
    % start small
    % # parameters vs # data
    % halving width
    % excessive depth
    % End-to-end.
    % dropout
\end{frame}

\begin{frame}
    \frametitle{Data}
    % normalize data
    % Extrapolation.
\end{frame}

\begin{frame}
    \frametitle{Engaging with the scientific community}
    % Obsolete papers.
    % Interpretability criticisms
\end{frame}

%%% Local Variables:
%%% mode: latex
%%% TeX-master: "../nn"
%%% End:
